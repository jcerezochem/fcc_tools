\documentclass[a4paper,11pt]{article}
\usepackage[utf8]{inputenc}

%opening
\title{\texttt{fcc\_analyzer}: a tool to visualize \fcc\ output}
\date{\textsc{Version: 0.2}\\\today}
\author{Javier Cerezo\\\texttt{j.cerezo@pi.iccom.cnr.it}}

% Margins
\textheight 22.0cm \textwidth 14.7cm \oddsidemargin 1cm
\evensidemargin 1cm \topmargin -0.25cm

%Packages
\usepackage{graphics}
\usepackage[pdftex]{graphicx}

\usepackage{amsmath}
\usepackage{amssymb}
\usepackage{color}
\usepackage{xcolor}
\usepackage{listings}
  \usepackage{courier}
 \lstset{
         basicstyle=\footnotesize\ttfamily, % Standardschrift
         %numbers=left,               % Ort der Zeilennummern
         numberstyle=\tiny,          % Stil der Zeilennummern
         %stepnumber=2,               % Abstand zwischen den Zeilennummern
         numbersep=5pt,              % Abstand der Nummern zum Text
         tabsize=2,                  % Groesse von Tabs
         extendedchars=true,         %
         breaklines=true,            % Zeilen werden Umgebrochen
         keywordstyle=\color{red},
            frame=b,         
 %        keywordstyle=[1]\textbf,    % Stil der Keywords
 %        keywordstyle=[2]\textbf,    %
 %        keywordstyle=[3]\textbf,    %
 %        keywordstyle=[4]\textbf,   \sqrt{\sqrt{}} %
         stringstyle=\color{white}\ttfamily, % Farbe der String
         showspaces=false,           % Leerzeichen anzeigen ?
         showtabs=false,             % Tabs anzeigen ?
         xleftmargin=17pt,
         framexleftmargin=17pt,
         framexrightmargin=5pt,
         framexbottommargin=4pt,
         %backgroundcolor=\color{lightgray},
         showstringspaces=false      % Leerzeichen in Strings anzeigen ?        
 }
 \lstloadlanguages{% Check Dokumentation for further languages ...
         %[Visual]Basic
         %Pascal
         %C
         %C++
         %XML
         %HTML
         Java
 }
    %\DeclareCaptionFont{blue}{\color{blue}} 

  %\captionsetup[lstlisting]{singlelinecheck=false, labelfont={blue}, textfont={blue}}
  \usepackage{caption}
\DeclareCaptionFont{white}{\color{white}}
\DeclareCaptionFormat{listing}{\colorbox[cmyk]{0.43, 0.35, 0.35,0.01}{\parbox{\textwidth}{\hspace{15pt}#1#2#3}}}
\captionsetup[lstlisting]{format=listing,labelfont=white,textfont=white, singlelinecheck=false, margin=0pt, font={bf,footnotesize}}

\begin{document}
\newcommand{\fcc}{$\mathcal{FC}$\textit{classes}}

\maketitle

\section{Description}
\texttt{fcc\_analyzer} is a python/matplotlib based tool that reads the output from \fcc\ (for a TI calculation), namely fort.21 (assignments) and fort.18 (spectrum), providing a graphical interface to get the assignments of the each individual transition.

\section{Installation}

\subsection{Local python installation}
No installation is required, but a proper version of the python interpreter along with numpy and matplotlib modules is needed. The program was developed using the following version of each element:

\begin{itemize}
 \item \texttt{python-2.7.5}\footnote{This program will \textbf{not} work with python-3.}
 \item \texttt{numpy-1.9.1}
 \item \texttt{matplotlib-1.4.2}
\end{itemize}

In most distributions, python-2.7 is already included (and is the default), but the modules numpy and matplotlib should be installed. Both modules are already available from Ubuntu or Fedora repositories (among many others), through either:

\begin{minipage}{0.5\textwidth}
 \begin{lstlisting}[label=ubuntu_repo,caption=\texttt{matplotlib} from Ubuntu]
apt-get install python-matplotlib
 \end{lstlisting}
\end{minipage}
\hspace*{0.1cm}
\begin{minipage}{0.45\textwidth}
  \begin{lstlisting}[label=fedora_repo,caption=\texttt{matplotlib} from Fedora]
yum install python-matplotlib
 \end{lstlisting}
\end{minipage}

However, these packaged versions might not be recent enough, and the script may behave wrongly\footnote{Attempts with \texttt{matplotlib-1.3.1} work, but with some capabilities disabled.}. Therefore, it is advisable to install them from the python setuptools (\texttt{pip}). Below, the installation steps followed in \textit{fresh}\footnote{Default installation plus the compilers (gcc, gfortran and g++).} Ubuntu 14.04 and Fedora 21 distributions are described.

\begin{minipage}{0.46\textwidth}
 \begin{lstlisting}[label=ubuntu_install,caption=Installing \texttt{matplotlib} in Ubuntu 14.04.]
#Install python-pip package
apt-get install python-pip
#Install python-dev package
apt-get install python-dev
#Install numpy
pip install numpy
#freetype and png required
apt-get install libfreetype6-dev
#Install matplotlib
pip install matplotlib
 \end{lstlisting}
\end{minipage}
\hspace*{0.1cm}
\begin{minipage}{0.46\textwidth}
  \begin{lstlisting}[label=fedora_install,caption=Installing \texttt{matplotlib} in Fedora 21.]
#Install python-pip package
yum install python-pip
#Install python-devel package
yum install python-devel
#Install numpy
pip install numpy
#freetype and png required
yum install freetype-devel
#Install matplotlib
pip install numpy
 \end{lstlisting}
\end{minipage}

After the above, you should have \texttt{matplotlib-1.4.2} (or above). If not, please, check the documentation of matplotlib, numpy and python in their respective websites. You can use \texttt{pip freeze} to see the version of all python modules installed.

\subsection{Self-contained application}
A binary produced with\\
\texttt{pyinstaller fcc\_analyzer.py --onefile}\\ is also provided. This contains, in principle, all python modules incorporated and 
would work on a machine without python installed. In practice, some dependencies make it not working for some cases, but one can give it a try.

\section{Using \texttt{fcc\_analyzer}}
After running a TI calculation with \fcc, and on the same folder, type (providing \texttt{fcc\_analyzer.py} is in your \texttt{PATH}):
\vspace*{0.2cm}

\texttt{fcc\_analyzer.py}
\vspace*{0.2cm}

\noindent This will open a \texttt{matplotlib} plot with all the transition divided by \textit{classes} (read from fort.21) and the final convoluted spectrum (read from fort.18). The following operations are possible:

\subsection{Identifying transitions}
\begin{itemize}
 \item \textbf{Right-mouse click over a transition}: show assignment of the highlighted transition. The rest of the info for this transition on fort.21 is shown on console.
 \item \textbf{Press \texttt{'+'} or \texttt{'-'}}: browse over the transitions, highlighting the following (\texttt{'+'}) or previous (\texttt{'-'}) one.
  \item \textbf{Left-mouse click over the legend items (classes)}: hide the class on the plot.
\end{itemize}

\subsection{Managing labels}
\begin{itemize}
 \item \textbf{Left-mouse click over a transition}: place a label over the transition showing the final modes.
 \item \textbf{Right-mouse click over a label (and hold)}: move the label. The pointer line will follow the label.
 \item \textbf{Left-mouse click over a label}: erase the label
\end{itemize}

\subsection{Cleaning}
\begin{itemize}
 \item \textbf{Right-mouse click over the plot title}: erase all labels and transition info.
 \item \textbf{Left-mouse click over a plot title}: erase transition info only.
\end{itemize}

\subsection{Using matplotlib backend functionalities\footnote{The availability of these options (and the aspect of the buttons) depends on the backend you are using for matplotlib. The ones shown here correspond to \textit{Qt4Agg} backend, which requires PyQt4 module (in Ubuntu and Fedora, it can be installed from the \texttt{python-qt4-dev} or \texttt{PyQt4-devel} packages.)}}
\begin{itemize}
 \item \textbf{Press \texttt{'o'} or zoom button, \includegraphics[width=0.5cm]{figs/butt_zoom.jpg}}: zoom into (left mouse click) or out (right mouse click) a rectangle.
 \item \textbf{Press \texttt{'p'} or move button, \includegraphics[width=0.5cm]{figs/butt_move.jpg}}: move the spectrum (left mouse click) or zoom (right mouse click).
 \item \textbf{Press save button, \includegraphics[width=0.5cm]{figs/butt_save.jpg}}: export image.
 \item \textbf{Press edit button, \includegraphics[width=0.5cm]{figs/butt_edit.jpg}}: customize, among others, the plot title.
\end{itemize}
(Note: mouse-click interactions with the plot are not active in zoom or move modes).

\subsection{Export plot to xmgrace}
\begin{itemize}
 \item \textbf{Central-mouse click over the plot title}: this will create a xmgrace file called \texttt{fcc\_analyzer.agr}, containing the convoluted and stick spectra and the labels selected in the plot. The axis ranges are kept. Note that the resulting file starts with the grace version number, which may be different in your case. The version number can be obtained by running:

\begin{minipage}{1.0\textwidth}
 \begin{lstlisting}[label=get_graceversion,caption=Getting \texttt{grace} version]
xmgrace -version 2>/dev/null | grep Grace- | sed "s/\./0/" | sed "s/\.//" | sed "s/Grace-//"
 \end{lstlisting}
\end{minipage}

\end{itemize}

\section{Error messages}

\begin{itemize}
 \item \texttt{ERROR: Check 0-0 transition}

The program could not read the intensity for the 0-0 transition on fort.21. In old \fcc\ versions, there was a bug on the format for this value on fort.21, and it is not printed if it is over 10. You have to insert it manually below \texttt{SPECTRUM} on the \texttt{0-0 TRANSITION} section of the file. The actual value can be taken from fort.8. 
\end{itemize}

Any other error might be a result of a wrong input file (the sanity checks are generally performed). In case of getting an error with ``good'' output files, feel free to send my a mail to: \texttt{j.cerezo@pi.iccom.cnr.it}

\end{document}
