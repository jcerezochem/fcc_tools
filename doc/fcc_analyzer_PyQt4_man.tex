\documentclass[a4paper,11pt]{article}
\usepackage[T1]{fontenc}

%opening
\title{\texttt{fcc\_analyzer\_PyQt4}\\ a GUI to visualize \fcc\ output}
\date{\textsc{Version: 0.1}\\\today}
\author{Javier Cerezo\\\href{mailto:j.cerezo@pi.iccom.cnr.it}{\texttt{j.cerezo@pi.iccom.cnr.it}}}

% Margins
\textheight 22.0cm \textwidth 14.7cm \oddsidemargin 1cm
\evensidemargin 1cm \topmargin -0.25cm

%Packages
\usepackage{graphics}
\usepackage[pdftex]{graphicx}

\usepackage{amsmath}
\usepackage{amssymb}
\usepackage{color}
\usepackage{xcolor}

\usepackage{hyperref}
% The following metadata will show up in the PDF properties
\hypersetup{
  colorlinks = true,
  urlcolor = blue,
}


\begin{document}
\setlength{\parskip}{0.5em}
\newcommand{\fcc}{$\mathcal{FC}$\textit{classes}}

\maketitle

\section{Description}
\texttt{fcc\_analyzer\_PyQt4} is a graphical user interface (GUI) written in python and relying on the PyQt4 and matplotlib libraries. It reads the output from \fcc\ (for a TI calculation), namely fort.21 (assignments) and either fort.22 (spectral histogram) or fort.18 (final convoluted spectrum), and opens an interactive plot that facilitates the analysis of the transitions, the investigation of different broadening schemes (if fort.22 is available) and the comparison with reference spectral data.

\section{Installation}

\subsection{Precompiled binary}
System-specific binaries are generated with \texttt{cx-Freeze} or \texttt{pyinstaller}. Currently, the following architechtures are supported (the binaries are located in \texttt{bin/<Arch>} folder):

\begin{itemize}
 \item \texttt{Linux64}: 64-bit Linux systems. The GLIBC version $\geq$2.14 is required. The folder contains the shared libraries (\texttt{.so}) along with the executable (\texttt{fcc\_analyzer\_PyQt4}). Generated with \texttt{cx-Freeze}.
 \item \texttt{Win64}: 64-bit Windows systems. The folder contains a one-file executable (\texttt{fcc\_analyzer\_PyQt4.exe}). Generated with \texttt{pyinstaller}. 
\end{itemize}

\subsection{Python script}
The source can be run directly as a python script from any computer architechture where a python interpreter is available. The python script is located in \texttt{src/}. \texttt{python} version 2.7 is required along with following modules (the minimum version tested is specified):

\begin{itemize}
 \item \texttt{matplotlib} (version $\geq1.4.2$)
 \item \texttt{numpy} (version $\geq1.9.1$)
 \item \texttt{PyQt4} (version $\geq4.10.4$)
\end{itemize}

% \clearpage

One of the most straightfoward methods to get a working python environment is through the \href{http://conda.pydata.org/docs/intro.html}{\texttt{conda}} package manager, which allows to install and manage different python versions easily. Among the advantages, it can be installed at the user folders/account not requiring admin (or root) privileges and it is available for different operating systems as a simple installer. A very suitable package for conda is miniconda, a light-weight version that can be downloaded from \href{http://conda.pydata.org/miniconda.html}{\texttt{http://conda.pydata.org/miniconda.html}}.

Once installed, the required modules to run the \texttt{fcc\_analyzer\_PyQt4} script can be installed through the command line (under any of the supported operating systems) by simply typing:\\

\texttt{conda install pyqt numpy matplotlib python=2}\\

The user is refered to the oficial \texttt{conda} documentation for further details.

% \clearpage

\section{Running the application}

\subsection{Launch the application}
The application should be invoked in the folder were the fort.21 and either fort.22 or fort.18 files that will be analysed reside, using the following syntax from the command line:\\

\texttt{fcc\_analyzer\_PyQt4 [flags]}\\

If no fort.21 is found in the current directory, a dialog to set the path to the files will be opened (this is useful when the program is called by mouser-clicking, e.g. in Windows). The following optional flags can be used when called from the command line:

\begin{tabular}{lll}
 \texttt{-type (abs|emi|ecd|cpl)}     && \begin{minipage}[t]{0.65\textwidth}
                                          Type of {\fcc} calculation performed.
                                         \end{minipage}\\\\
 \texttt{-maxC N}                     && \begin{minipage}[t]{0.65\textwidth}
                                          Maximum class to by loaded (\texttt{N}$\leq7$).
                                         \end{minipage}\\\\
 \texttt{-h}                          && \begin{minipage}[t]{0.65\textwidth}
                                          Show help.
                                         \end{minipage}\\\\
\end{tabular}

\clearpage

\subsection{Using the application}
The GUI consists of the following parts:

\begin{figure}[h!]
\begin{center}
  \includegraphics[width=15cm]{figs/fcc_analyzer_screeshot.png}
\end{center}
\caption{Screenshot of \texttt{fcc\_analyzer\_PyQt4} highlighting the different parts.}
\end{figure}

The interaction and analysis of the simulated spectrum are performed with the different widgets in the plot. Below, the different parts are described.

\subsubsection{Menu Box}
\begin{itemize}
 \item File
 \begin{itemize}
  \item Save plot: save plot to a png file.
  \item Export to xmgrace: export plot data (including labels) to xmgrace file. A export assistant is first raised to select how to organize the plots:
  \begin{itemize}
   \item Overlaid graphs: place the stick and spectra in different graphs, that are overlaid, as in the matplotlib plot.
   \item Same graph: include sticks and spectra in the same graph. The spectra are normalized to the maximum stick intensity.
  \end{itemize}

  \item Import plot: Load reference spectrum to the Spectrum Area. Units: (eV|cm$^{-1}$|nm).
  \item Quit: exit the application
 \end{itemize}

 \item Analyse
 \begin{itemize}
  \item Momenta: compute momenta of the convoluted and write to the Analysis Box.
 \end{itemize}

 \item Manipulate
 \begin{itemize}
  \item Shift to simulated: shift reference spectrum (if loaded) to match the convoluted spectrum.
  \item Scale to convoluted: scale reference spectrum (if loaded) to match the convoluted spectrum.
 \end{itemize}

 \item Help
 \begin{itemize}
  \item Instructions: show a short guide to use the application.
  \item About: show general info about the application.
 \end{itemize}
\end{itemize}
 
\subsubsection{Spectrum Area}
This area is interactive. The following actions are possible:
\begin{itemize}
 \item Right-mouse click on a stick: highlight and get info about the transition (written into Analysis Box).
 \item Left-mouse click on a stick: add label over the stick
 \item Right-mouse click on a label: drag the label
 \item Right-mouse click on a legend line: activate/deactivate plot
 \item Press ``+''/``-'' keys to browse back and forward between highlighted stick
\end{itemize}

\subsubsection{Analysis Box}
In this box, the information about the highlighted transitions is shown, as well as the analysis of the momenta. This box is not editable, but the info can be copied.

\subsubsection{Search Box}
Selector of transitions. The general syntax is the following:

{
\scriptsize
\texttt{Mode1'(Quanta1'),Mode2'(Quanta2')... {-}{-}> Mode1(Quanta1),Mode2(Quanta2)...}
}\\

where \texttt{Mode1'(Quanta1')} refer to mode \texttt{Mode1'} in the initial state that is excited \texttt{Quanta1'} quanta, and \texttt{Mode1(Quanta1)} are the equivalent entities for the final state.

The ground vibrational state is represented by a zero (\texttt{0}). When the initial state is in the ground state, the initial part, \texttt{0{-}{-}>} can be omitted. Some examples:

\begin{itemize}
 \item \texttt{8(1),9(2)}: select the transition from the ground initial state to the final state where mode 8 is excited 1 quantum and mode 9 is excited 2 quanta.
 \item \texttt{0{-}{-}>8(1),9(2)}: same as above
 \item \texttt{8(1){-}{-}>0}: select transition from initial state excited 1 quantum on mode 8 to the final ground state.
 \item \texttt{0{-}{-}>0}: select 0-0 transition
\end{itemize}

To select a progression in the final state, use \texttt{P} instead of the number of quanta. This keyword can only be used for one mode in the same selection. Some examples:

\begin{itemize}
 \item \texttt{8(p)}: select progression of mode 8 in the final state.
 \item \texttt{0{-}{-}>8(p),9(2)}: select progression of mode 8 in the final state when, simultaneously, mode 9 is excited 2 quanta.
 \item \texttt{1(1){-}{-}>8(p)}: select progression of mode 8 in the final state, starting from the initial state where mode 1 is excited one quantum.
\end{itemize}

\subsubsection{Plot Toolbar}
This is formed by two check boxes and the standard matplotlib toolbar with PyQt4 back-end. The check boxes are:
\begin{itemize}
 \item Fix y-axis: whether or not scale y-axis when the convolution or the data type are updated.
 \item Fix legend: whether the legend can be displaced with the mouse of not.
\end{itemize}

The matplotlib toolbar include:
\begin{itemize}
 \item \includegraphics[width=0.5cm]{figs/butt_zoom.jpg}: zoom into (left mouse click) or out (right mouse click) a rectangle.
 \item \includegraphics[width=0.5cm]{figs/butt_move.jpg}: move the spectrum (left mouse click) or zoom (right mouse click).
 \item \includegraphics[width=0.5cm]{figs/butt_save.jpg}: export image.
 \item \includegraphics[width=0.5cm]{figs/butt_edit.jpg}: customize axes properties.
\end{itemize}

\subsubsection{Cleaning Buttons}
\begin{itemize}
 \item Clean(Panel): remove highlight on transitions and clear the analysis box.
 \item Clean(Labels): clear all labels added to the spectrum
\end{itemize}

\subsubsection{Convolution Controls}
These controls are only active if the fort.22 file is available. Note that this is not present for old versions of \fcc.
\begin{itemize}
 \item Broadening selector [Gau|Lor]: type of broadening function
 \item HWHM slider and box: set the HWHM of the broadening function. The box is editable and allows a more precise selection of the value, even beyond the slider limits (0.01 to 0.1\,eV).
 \item Input bins check box: use the bins given in the input fort.22 file. Otherwise, a new histogram with only 1000 bins is used (which is faster).
\end{itemize}

\subsubsection{Data Type Selector}
Select the type of spectrum: [Intensity|Lineshape].

\subsubsection{Ref. Spectrum Table}
Information about the reference spectrum (loaded with File->Import plot).

\begin{itemize}
 \item Loaded?: whether reference spectrum is loaded. If so the file name is shown. The red button with the cross deletes the reference spectrum.
 \item Shift(eV): shift applied to the reference spectrum (in eV). This box is editable. The value always refer to the originally loaded value, unless the blue T button is pressed, which resets the reference value to the current one.
 \item Y-scale: scaling applied to the reference spectrum Y-axis. This box is editable. The value always refer to the originally loaded value, unless the blue T button is pressed, which resets the reference value to the current one.
\end{itemize}


\end{document}
